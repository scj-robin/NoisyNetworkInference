%!TEX root = VEMScoreEdges.tex



\subsection{Review of the literature}
Notes on identifiability based on papers : 

\begin{itemize}
\item \cite{AMR09} : "Allman, Elizabeth S. and Matias, Catherine and Rhodes, John A." : \emph{Identifiability of parameters in latent structure models with many observed variables}
\item \cite{AMR10} "Allman, Elizabeth S. and Matias, Catherine and Rhodes, John A." :  \emph{Parameter identifiability in a class of random graph mixture models}
\item \cite{T63}"Teicher, Henry":    \emph{Identifiability of Finite Mixtures}
\item \cite{T67} "Teicher, Henry":    \emph{Identifiability of Mixtures of product measures}
\end{itemize}

\vspace{2em}


\paragraph{What is done in    \cite{AMR10}}: identifiability in weighted SBM 
\begin{eqnarray*}
S_{ij} | Z_i =k ,Z_j = \ell &\sim& \mu_{k\ell}\\
  \mu_{k\ell} &=& (1-\gamma_{k\ell})  \delta_ {\{0\}} + + \gamma_{kl} F_{k\ell}( \cdot)
\end{eqnarray*}
 for \underline{uni dimensional $S$ and symmetric}  with 
\begin{itemize} 
\item \underline{ $ F_{k\ell}( \cdot)$ parametric} (Theorem 12 of \cite{AMR10})  : $ F( \cdot; \theta_{k\ell})$ under the following assumptions: 
\begin{enumerate}
\item[[A1]]  The $K(K+1)/2$ parameter values $\theta_{k\ell}$ are distinct
\item[[A2]]  The family of measures $\Mcal = \{ F( \cdot; \theta) | \theta\in\Theta\}$  is such that 
\begin{itemize}
\item[[A2](i)] all elements of $\Mcal$ have no point mass at $0$ 
\item[[A2](ii)] the parameters of finite mixtures of  measures of $\Mcal$ are identifiable (up to label switching) i.e.
$$ \sum_{m=1}^M \alpha_m F(\cdots, \theta_m) =  \sum_{m=1}^M \alpha'_m F(\cdots, \theta'_m) \Rightarrow  \sum_{m=1}^M \alpha_m \delta_{\theta_m} =  \sum_{m=1}^M \alpha'_m \delta_{\theta'_m} $$ 
\end{itemize}

In particular : true for Gaussian (\cite{T63}) and Laplace. 
\end{enumerate}
\item \underline{ $ F_{k\ell}( \cdot)$ non-parametric} (Theorem 14 of \cite{AMR10}) : if the $\mu_{k \ell}$ are \emph{linearly independent} (to be detailed)
\end{itemize}


\paragraph{About the demonstrations} 

\begin{itemize}
\item  \emph{Parametric case}  It is done from the distribution of a triplet ($S_{ij},S_{ik}, S_{jk}$) and using \cite{T63}. How to adapt it  to our case? %because only relies on \cite{T63} and then it is really simple. 
\item  \emph{Nonparametric case} : only depends on the linear independancy of the $\mu_{k\ell}$. We have to precise it for our case? 
\end{itemize}


\subsection{Proof in the parametric uni-dimensional context}

I tried to mimic/extend the proof of \cite{AMR10}  but I don't think we are in the same scope.  

\paragraph{Distribution of the $S_{ij}$}

 \begin{align*}
 \mathbb{P}(S_{ij}) &= \sum_{q,\ell} \pi_{q} \pi_{\ell} [ (1-\gamma_{q\ell})F_0(S_{ij}) + \gamma_{q\ell} F_1(S_{ij} )]\\
&= \left[1-  \sum_{q, \ell}  \pi_\ell \pi_q  \gamma_{q,\ell} \right] F_{0}(S_{ij}) +   \left[ \sum_{q, \ell}  \pi_q \pi_\ell    \gamma_{q,\ell} \right] F_{1}(S_{ij})\\
 \end{align*}
So \textbf{assuming that $F_0$ and $F_1$ are such that any mixture of those two distributions is identifiable}, we obtain the identifiability of $\theta_0$, $\theta_1$ and $ \sum_{q, \ell}  \pi_{\ell} \pi_q   \gamma_{q,\ell}$. 


So we have identifiability of $\pi'  \gamma  \pi$. 
It seems to me that once we have identified $\theta_0$ and $\theta_1$ we will be able to apply to proof of Célisse \& al. \cite{CDP12}, which is the one I know better. Which is the thing you said : meaning that once we have identified to high level, we are identifiable just like any binary SBM. 







%\paragraph{Distribution of the triplet  ($S_{ij},S_{ik}, S_{jk}$)}
%
% \begin{align*}
% \mathbb{P}(S_{ij},S_{ik}, S_{jk}) &= \sum_{q,\ell,m} \pi_{q} \pi_{\ell} \pi_{m} [ (1-\gamma_{q\ell})F_0(S_{ij}) + \gamma_{q\ell} F_1(S_{ij} )] [(1-\gamma_{qm})F_0(S_{ik}) + \gamma_{qm} F_1(S_{ik})]\\
%&   [ (1-\gamma_{\ell m})F_0(S_{jk}) + \gamma_{\ell m} F_1(S_{jk})]\\
%&= \sum_{q, \ell,m} \sum_{(u,v,w) \in \{0,1\}^3}  \eta_{q,\ell,m,u,v} F_{u}(S_{ij}) F_{v}(S_{ik}) F_{w}(S_{jk})\\
%&= \sum_{(u,v,w) \in \{0,1\}^3}  \left(\sum_{q, \ell,m} \eta_{q,\ell,m,u,v}\right) F_{u}(S_{ij}) F_{v}(S_{ik}) F_{w}(S_{jk})\\
%&=  \sum_{(u,v,w) \in \{0,1\}^3}  \left(\sum_{q, \ell,m} \eta_{q,\ell,m,u,v}\right) F_{u,v,w}(S_{ij},S_{ik},S_{jk})
%\end{align*}
%with $$\eta_{q,\ell,m,u,v} = \pi_{q} \pi_{\ell} \pi_{m} (1 - \gamma_{q\ell})^{1-u}   \gamma_{q\ell} ^{u}  (1 - \gamma_{q\ell})^{1-u}   \gamma_{q\ell} ^{u}  (1 - \gamma_{qm})^{1-v}   \gamma_{qm} ^{v} (1 - \gamma_{\ell m })^{1-w}   \gamma_{\ell m } ^{w}.$$  \\
%The distribution of $(S_{ij},S_{ik}, S_{jk})$ is a mixture  (weights  = $ \sum_{q, \ell,m} \eta_{q,\ell,m,u,v}$) of   the following distributions 
%$$ F (s) =  F_u(s_1,\theta_u) F_v(s_1,\theta_v)  F_w(s_1,\theta_w) $$ 
% where $F \in \Fcal$   with 
%$$ \Fcal  = \{ F (s; \theta_0, \theta_1)  :  F (s; \theta_0, \theta_1)  = F_u(s_1,\theta_u),F_v(s_2,\theta_v) F_w(s_3,\theta_w) , (u,v,w) \in \{0,1\}^3,  \theta_0, \in \Theta_0, \theta_1 \in \Theta_1\}$$
% 
% 
%\noindent \textbf{Asumptions};  
% \noindent  [A1]  we assume that any mixtures of elements of $\Fcal$ is identifiable.  (to develop to get assumptions on $F_0$ and $F_1$). 
%
% 
%
%The, under assumption [A1], we have : 
%
%
%
%Then using Theorem 1 of \cite{T67} we have the identifiability of any mixture of the  product  measures. 
%











